\documentclass[a4paper]{article}
\usepackage[margin=1in]{geometry}

\title{Math Notes}
\date{2018-03-15}
\author{WANG Weisheng}

\begin{document}
\maketitle
\newpage

\section{Analyse}
\subsection{Simpson's rule}
si $\mathcal{Q}(t)$ est un polynom d'ordre 3, alors on a:
\begin{equation}
\int_{a}^{b}\mathcal{Q}(t)dt=\frac{b-a}{6}\big[\mathcal{Q}(a)+4\mathcal{Q}(\frac{a+b}{2})+\mathcal{Q}(b)\big]
\end{equation}

\section{Factorization}
\begin{itemize}
  \item Utile pour prouver l'inegalite de bernouille.
  \begin{equation}
  (x^n-y^n)=(x-y)(x^{n-1}+x^{n-2}y+x^{n-3}y^{2}+\cdots+xy^{n-2}+y^{n-1})
  \end{equation}

  \item The text in the entries may be of any length.
\end{itemize}


\end{document}
